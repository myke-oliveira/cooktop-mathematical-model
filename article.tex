\documentclass[
	% -- opções da classe memoir --
	article,			% indica que é um artigo acadêmico
	11pt,				% tamanho da fonte
	oneside,			% para impressão apenas no recto. Oposto a twoside
	a4paper,			% tamanho do papel. 
	% -- opções da classe abntex2 --
	%chapter=TITLE,		% títulos de capítulos convertidos em letras maiúsculas
	%section=TITLE,		% títulos de seções convertidos em letras maiúsculas
	%subsection=TITLE,	% títulos de subseções convertidos em letras maiúsculas
	%subsubsection=TITLE % títulos de subsubseções convertidos em letras maiúsculas
	% -- opções do pacote babel --
	english,			% idioma adicional para hifenização
	brazil,				% o último idioma é o principal do documento
	sumario=tradicional
	]{abntex2}


% ---
% PACOTES
% ---

% ---
% Pacotes fundamentais 
% ---
\usepackage{lmodern}			% Usa a fonte Latin Modern
\usepackage[T1]{fontenc}		% Selecao de codigos de fonte.
\usepackage[utf8]{inputenc}		% Codificacao do documento (conversão automática dos acentos)
\usepackage{indentfirst}		% Indenta o primeiro parágrafo de cada seção.
\usepackage{nomencl} 			% Lista de simbolos
\usepackage{color}				% Controle das cores
\usepackage{graphicx}			% Inclusão de gráficos
\usepackage{microtype} 			% para melhorias de justificação
\usepackage{booktabs}           % tabelas
% ---
		
% ---
% Pacotes adicionais, usados apenas no âmbito do Modelo Canônico do abnteX2
% ---
\usepackage{lipsum}				% para geração de dummy text
% ---
		
% ---
% Pacotes de citações
% ---
\usepackage[brazilian,hyperpageref]{backref}	 % Paginas com as citações na bibl
\usepackage[alf]{abntex2cite}	% Citações padrão ABNT
% ---

% ---
% Configurações do pacote backref
% Usado sem a opção hyperpageref de backref
\renewcommand{\backrefpagesname}{Citado na(s) página(s):~}
% Texto padrão antes do número das páginas
\renewcommand{\backref}{}
% Define os textos da citação
\renewcommand*{\backrefalt}[4]{
	\ifcase #1 %
		Nenhuma citação no texto.%
	\or
		Citado na página #2.%
	\else
		Citado #1 vezes nas páginas #2.%
	\fi}%
% ---

% --- Informações de dados para CAPA e FOLHA DE ROSTO ---
\title{\Huge{Análise de um Modelo Matemático para um Cooktop Indutivo}}

\author{Myke Albuquerque Pinto de Oliveira}

\local{Brasil}
\data{2019, 0.0}
% ---

% ---
% Configurações de aparência do PDF final

% alterando o aspecto da cor azul
\definecolor{blue}{RGB}{41,5,195}

% informações do PDF
\makeatletter
\hypersetup{
     	%pagebackref=true,
		pdftitle={\@title}, 
		pdfauthor={\@author},
    	pdfsubject={Modelo Matemático de um Cooktop Indutivo},
	    pdfcreator={LaTeX with abnTeX2},
		pdfkeywords={cooktop}{indução}{eletromagnetismo}{modelagem matemática}{física}, 
		colorlinks=true,       		% false: boxed links; true: colored links
    	linkcolor=blue,          	% color of internal links
    	citecolor=blue,        		% color of links to bibliography
    	filecolor=magenta,      		% color of file links
		urlcolor=blue,
		bookmarksdepth=4
}
\makeatother
% --- 

% ---
% compila o indice
% ---
\makeindex
% ---

% ---
% Altera as margens padrões
% ---
\setlrmarginsandblock{3cm}{3cm}{*}
\setulmarginsandblock{3cm}{3cm}{*}
\checkandfixthelayout
% ---

% --- 
% Espaçamentos entre linhas e parágrafos 
% --- 

% O tamanho do parágrafo é dado por:
\setlength{\parindent}{1.3cm}

% Controle do espaçamento entre um parágrafo e outro:
\setlength{\parskip}{0.2cm}  % tente também \onelineskip

% Espaçamento simples
\SingleSpacing


% ----
% Início do documento
% ----
\begin{document}

% Seleciona o idioma do documento (conforme pacotes do babel)
%\selectlanguage{english}
\selectlanguage{brazil}

% Retira espaço extra obsoleto entre as frases.
\frenchspacing 

% ----------------------------------------------------------
% ELEMENTOS PRÉ-TEXTUAIS
% ----------------------------------------------------------

%---
%
% Se desejar escrever o artigo em duas colunas, descomente a linha abaixo
% e a linha com o texto ``FIM DE ARTIGO EM DUAS COLUNAS''.
% \twocolumn[    		% INICIO DE ARTIGO EM DUAS COLUNAS
%
%---

% página de titulo principal (obrigatório)
\maketitle


% titulo em outro idioma (opcional)



% resumo em português
\begin{resumoumacoluna}
 Conforme a ABNT NBR 6022:2018, o resumo no idioma do documento é elemento obrigatório. 
 Constituído de uma sequência de frases concisas e objetivas e não de uma 
 simples enumeração de tópicos, não ultrapassando 250 palavras, seguido, logo 
 abaixo, das palavras representativas do conteúdo do trabalho, isto é, 
 palavras-chave e/ou descritores, conforme a NBR 6028. (\ldots) As 
 palavras-chave devem figurar logo abaixo do resumo, antecedidas da expressão 
 Palavras-chave:, separadas entre si por ponto e finalizadas também por ponto.
 
 \vspace{\onelineskip}
 
 \noindent
 \textbf{Palavras-chave}: cooktop, indução, eletromagnetismo, modelagem matemática, física.
\end{resumoumacoluna}


% resumo em inglês
\renewcommand{\resumoname}{Abstract}
\begin{resumoumacoluna}
 \begin{otherlanguage*}{english}
   According to ABNT NBR 6022:2018, an abstract in foreign language is optional.

   \vspace{\onelineskip}
 
   \noindent
   \textbf{Keywords}: cook-top, induction, electromagnetism, mathematical model, physics.
 \end{otherlanguage*}  
\end{resumoumacoluna}

% ]  				% FIM DE ARTIGO EM DUAS COLUNAS
% ---

\begin{center}\smaller
\textbf{Data de submissão e aprovação}: elemento obrigatório. Indicar dia, mês e ano

\textbf{Identificação e disponibilidade}: elemento opcional. Pode ser indicado 
o endereço eletrônico, DOI, suportes e outras informações relativas ao acesso.
\end{center}

% ----------------------------------------------------------
% ELEMENTOS TEXTUAIS
% ----------------------------------------------------------
\textual

% ----------------------------------------------------------
% Introdução
% ----------------------------------------------------------
\section{Introdução}

Um cooktop de indução (informalmente fogão de indução) é um equipamento que substitui o fogão no cozimento de alimentos. O cooktop de indução difere do cooktop a gás porque sua fonte de energia é a eletricidade e não o gás liquefeito de petróleo (GLP), e também difere do cooktop elétrico por resistência, pois ao invés de possuir uma superfície metálica aquecida por uma resistência elétrica, o cooktop de indução não apresenta nenhuma superfície aquecida além da própria panela onde o alimento está sendo preparado. O princípio de funcionamento do cooktop de indução é baseado na geração de um campo magnético variável através de uma bobina espiralada e a geração de um campo elétrico induzido no material da panela, esse campo elétrico, por sua vez, gera correntes elétricas na panela e calor por efeito Joule.

Entre as vantangens de se adotar um cooktop de indução, pode-se citar:

\begin{itemize}
	\item Dispensa o uso de botijão de gás.
	\item Não há superfícies quentes no equipamento e não funciona sem a presença da panela no equipamento.
	\item Melhor eficiência no aquecimento, pois não perde calor na transferência por convecção.
	\item A maioria dos modelos é microcontrolada e fornece opções como o desligamento temporizado e o controle automático de potência.
\end{itemize}

O presente artigo é o resultado de discussão e pesquisa da Unidade de Aprendizagem "Resolução de Problemas no Contexto da Física" e tem como objetivo estabelecer a relação entre a corrente elétrica que circula na bobina espiralada do cooktop de indução e a potência calórica fornecida. Incluindo essa introdução, o artigo está dividido em cinco seções, a segunda seção apresenta mais detalhes sobre o problema a ser resolvido. A terceira seção resolve detalhadamente o problema utilizando técnicas numéricas e a linguagem de programação Python 3. A quarta seção encerra este artigo sob a ótica educativa e de formação de competências.

\section{Problema de estudo}

Nesta seção, são apresentados todos os detalhes necessários para resolver os problema, todas as grandezas físicas envolvidas são definidas e as variáveis relevantes para o problema são escolhidas.

\subsection{Enunciado do problema}

Dado um condutor elétrico $ L $ de formato espiralado percorrido por uma corrente $ i(t) $ e uma chapa $ C $ sobre esse condutor, conforme figura \ref{fig:esquema}. Estabelecer a relação entre a corrente $ i(t) $ e a potência de calor $ P(t) $ gerada na chapa metálica.

\begin{figure}[h]
	\centering
	\includegraphics[width=0.7\linewidth]{figures/fig1}
	\caption[Esquema de um cooktop de indução]{Esquema de um cooktop de indução}
	\label{fig:esquema}
\end{figure}

\subsection{Tema que contextualiza o problema}

Este problema se baseia na área da física e engenharia elétrica de eletromagnetismo.

\subsection{Objetos de Estudos}

Seguem a definição das grandezas físicas que são envolvidas pelo fenômeno.

\textbf{Corrente elétrica:} é o movimento em uma direção preferencial de portadores de carga (elétrons ou íons), ou seja, uma taxa de transferência de cargas elétricas. A unidade de corrente elétrica no SI é o Ampère ($ A $), ou Coulomb por segundo ($ C/s $). \cite{keller:1998}

\textbf{Campo elétrico:} é uma grandeza vetorial associada a posição que estabelece a relação entre a força elétrica exercida sobre uma partícula e sua carga elétrica. O campo elétrico pode ser provocado pela presença de outras cargas elétricas ou induzida por campos magnéticos variantes no tempo. A unidade de campo elétrico no SI é o Newton por Coulomb ($ N/C $) ou Volt por metro ($ V/m $). \cite{keller:1998}

\textbf{Campo magnético:} é uma grandeza vetorial associada a posição que estabelece a relação entre a força magnética exercida sobre uma carga elétrica em movimento. O campo magnético é gerado por cargas elétricas em movimento (também denominadas elementos de corrente elétrica) e também pode ser induzido por campos elétricos variantes no tempo. A unidade de campo magnético no SI é o Newton-segundo por Coulomb-metro ($ (Ns)/(Cm) $), Newton por Ampère-metro ($ N/(Am) $) ou Tesla ($ T $). \cite{keller:1998}

\textbf{Calor:} Energia térmica em transito, ou seja, transferida, gerada ou consumida. A unidade de energia térmica no SI é o Joule ($ J $).

\subsection{Variáveis e parâmetros do problema}

Parâmetros do problema são grandezas físicas dadas na definição do problema, ou valores de entrada para o cálculo, enquanto variáveis são grandezas físicas a serem determinadas durante a resolução do problema, A tabela \ref{tab:parametros} apresenta os parâmetros do problema, já a tabela \ref{tab:variaveis} apresenta as variáveis do problema. Por fim a tabela \ref{tab:constantes} apresenta constantes físicas que serão necessárias para a resolução do problema.

\begin{table}[h]
	\begin{tabular}{@{}llll@{}}
		\toprule
		\textbf{Parâmetro}                     & \textbf{Simbolo} & \textbf{Unidade de Medida} & \textbf{Restrição} \\ \midrule
		Raio da bobina                         & $ R $            & m (metro)                  & $ \ge 0 $          \\
		Número de espiras na bobina            & $ N $            & unidade                    & natural            \\
		Espaçamento entre a bobina e a panela  & $ d $            & m (metro)                  & $ \ge 0 $          \\
		Corrente elétrica que circula a bobina & $ i(t) $         & A (ampère)                 & função do tempo    \\ \bottomrule
	\end{tabular}
	\caption{Parâmetros do problema}
	\label{tab:parametros}
\end{table}

\begin{table}[h]
	\begin{tabular}{@{}llll@{}}
		\toprule
		\textbf{Variável}             & \textbf{Simbolo}            & \textbf{Unidade de Medida}        & \textbf{Restrição}     \\ \midrule
		Tempo (variável independente)  & $ t $                      & s (segundo)                       &                \\ 
		Coordenadas Cartesianas        & $ x, y, z $                & m (metro)                         &                \\ 
		Coordenadas Cilíndricas        & $ r, \theta $              & m (metro), rad (radiano)          &                \\ 
		Campo Elétrico                 & $ \textbf{E}(x, y, z, t) $ & V/m (Volt por metro)              & campo vetorial \\ 
		Campo Magnético                & $ \textbf{B}(x, y, z, t) $ & T (Tesla)                         & campo vetorial \\
		Densidade de corrente elétrica & $ \textbf{J}(x, y, z, t) $ & A/m2 (Ampére por metro quatrado)  & campo vetorial \\
		Potência calórica              & $ P(x, y, z, t) $          & W (Watt)                          & campo escalar  \\ \bottomrule
	\end{tabular}
	\caption{Variáveis do problema}
	\label{tab:variaveis}
\end{table}

\begin{table}[h]
	\begin{tabular}{@{}lll@{}}
		\toprule
		Constante      & Símbolo      & Valor \\ \midrule
		Pemissividade  & $ \epsilon $ & $ 8,85 \cdot 10^{-12} F m^{-1} $ \\
		Permeabilidade & $ \mu $      & $ 4\pi \cdot 10^-7 T m A^{-1}  $ \\
		Resistividade  & $ \varrho $     & $ \Omega m^{-1} $             \\ \bottomrule
	\end{tabular}
	\caption{Constantes físicas}
	\label{tab:constantes}
\end{table}

\section{Resolução detalhada do problema}

O primeiro passo para resolver o problema é analisar o percurso que a corrente elétrica realiza no equipamento. Foi dado que o condutor é uma espiral com os raio $ R $, $ N $ espiras e espaçamento radial entre as espiras $ e $. Devido a simetria radial do condutor, é bastante conveniente utilizar coordenadas cilíndricas, associando a cada trecho do condutor um raio $ r_L(\theta) $ e um angulo $ \theta $. A função $ r_l(\theta) $ é expressa de acordo com a equação \ref{eq:condutor-polar}.

\begin{equation} \label{eq:condutor-polar}
	\begin{array}{l}
		r_L(\theta) = R - \frac{e}{2\pi} \theta \\
		0 \le \theta \le 2\cdot N \cdot \pi
	\end{array}
\end{equation}

A figura \ref{fig:condutor-polar} apresenta o resultado da equação \ref{eq:condutor-polar}.

\begin{figure}[h]
	\centering
	\includegraphics[width=0.7\linewidth]{figures/espirapolar}
	\caption[Condutor em coordenadas polares]{Condutor em coordenadas polares}
	\label{fig:condutor-polar}
\end{figure}

A representação em coordenadas cartesianas é mais conveniente no momento de calcular o campo magnético em um ponto dado. A conversão para coordenadas cartesianas é realizada pela transformação \ref{eq:polar-cartesiano}.

\begin{equation} \label{eq:polar-cartesiano}
	\begin{array}{l}
	x = r \cdot \cos \theta \\ 
	y = r \cdot \sin \theta
	\end{array}
\end{equation}

Essa transformação nos permite reescrever curva do condutor da bobina usando as equações paramétricas \ref{eq:condutor-parametricas}.

\begin{equation} \label{eq:condutor-parametricas}
	\begin{array}{l}
		x(\theta) = \left(  R - \frac{e}{2\pi} \theta \right) \cdot \cos \theta \\
		y(\theta) = \left(  R - \frac{e}{2\pi} \theta \right) \cdot \sin \theta \\
		0 \le \theta \le 2\cdot N \cdot \pi
	\end{array}
\end{equation}

A figura \ref{fig:espiracartesiana} apresenta a mesma espira em coordenadas cartesianas.

\begin{figure}[h]
	\centering
	\includegraphics[width=0.7\linewidth]{figures/espiracartesiana}
	\caption[Condutor em coordenadas cartesianas]{Condutor em coordenadas cartesianas}
	\label{fig:espiracartesiana}
\end{figure}

O campo magnético induzido por essa bobina pode ser calculado usando a lei de Biot-Savart, equação \ref{eq:biot-savart}. No entando, usaremos uma versão discreta da lei de Biot-Savart, para tanto, é necessário discretizar os elementos de corrente na bobina. Esses elementos de corrente discretos estão representados na figura \ref{fig:elementosdecondutor}.

\begin{figure}[h]
	\centering
	\includegraphics[width=0.7\linewidth]{figures/elementosdecondutor}
	\caption[Elementos de corrente discretizados]{Elementos de corrente discretizados}
	\label{fig:elementosdecondutor}
\end{figure}

\begin{equation} \label{eq:biot-savart}
	d\textbf{B} = \frac{\mu_0}{4\pi} \frac{i(t)d\textbf{l} \times \hat{\textbf{r}}}{r^2}
\end{equation}

Enumerando os elementos discretizados na \ref{fig:elementosdecondutor}, e aplicando a equação \ref{eq:biot-savart-dicreta}, tem-se o resultado da figura \ref{fig:magnetico}. Foi efetuado o cálculo com uma corrente constante de $ 1 A $ em uma bobina de $ 0.05 m $, espaçamento $ 0.005 m $ e 10 voltas, discretizado em 300 elementos.

\begin{equation} \label{eq:biot-savart-dicreta}
	\textbf{B}(x, y, z, t) \approx \frac{\mu_0i(t)}{4\pi} \sum_{i = 1}^{m} \left( \frac{\Delta\textbf{l}_1 \times \hat{\textbf{r}_1}}{r_1^2} \right)
\end{equation}

\begin{figure}[h]
	\centering
	\includegraphics[width=0.7\linewidth]{figures/magnetico}
	\caption[Campo magnético gerado]{Campo magnético gerado}
	\label{fig:magnetico}
\end{figure}

\begin{equation}
	\textbf{r} (\theta) = 
\end{equation}


\section{Análise e considerações finais}



\bibliography{references}


\end{document}
