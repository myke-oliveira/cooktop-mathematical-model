%% abtex2-modelo-artigo.tex, v<VERSION> laurocesar
%% Copyright 2012-<COPYRIGHT_YEAR> by abnTeX2 group at http://www.abntex.net.br/ 
%%
%% This work may be distributed and/or modified under the
%% conditions of the LaTeX Project Public License, either version 1.3
%% of this license or (at your option) any later version.
%% The latest version of this license is in
%%   http://www.latex-project.org/lppl.txt
%% and version 1.3 or later is part of all distributions of LaTeX
%% version 2005/12/01 or later.
%%
%% This work has the LPPL maintenance status `maintained'.
%% 
%% The Current Maintainer of this work is the abnTeX2 team, led
%% by Lauro César Araujo. Further information are available on 
%% http://www.abntex.net.br/
%%
%% This work consists of the files abntex2-modelo-artigo.tex and
%% abntex2-modelo-references.bib
%%

% ------------------------------------------------------------------------
% ------------------------------------------------------------------------
% abnTeX2: Modelo de Artigo Acadêmico em conformidade com
% ABNT NBR 6022:2018: Informação e documentação - Artigo em publicação 
% periódica científica - Apresentação
% ------------------------------------------------------------------------
% ------------------------------------------------------------------------

\documentclass[
	% -- opções da classe memoir --
	article,			% indica que é um artigo acadêmico
	11pt,				% tamanho da fonte
	oneside,			% para impressão apenas no recto. Oposto a twoside
	a4paper,			% tamanho do papel. 
	% -- opções da classe abntex2 --
	%chapter=TITLE,		% títulos de capítulos convertidos em letras maiúsculas
	%section=TITLE,		% títulos de seções convertidos em letras maiúsculas
	%subsection=TITLE,	% títulos de subseções convertidos em letras maiúsculas
	%subsubsection=TITLE % títulos de subsubseções convertidos em letras maiúsculas
	% -- opções do pacote babel --
	english,			% idioma adicional para hifenização
	brazil,				% o último idioma é o principal do documento
	sumario=tradicional
	]{abntex2}


% ---
% PACOTES
% ---

% ---
% Pacotes fundamentais 
% ---
\usepackage{lmodern}			% Usa a fonte Latin Modern
\usepackage[T1]{fontenc}		% Selecao de codigos de fonte.
\usepackage[utf8]{inputenc}		% Codificacao do documento (conversão automática dos acentos)
\usepackage{indentfirst}		% Indenta o primeiro parágrafo de cada seção.
\usepackage{nomencl} 			% Lista de simbolos
\usepackage{color}				% Controle das cores
\usepackage{graphicx}			% Inclusão de gráficos
\usepackage{microtype} 			% para melhorias de justificação
% ---
		
% ---
% Pacotes adicionais, usados apenas no âmbito do Modelo Canônico do abnteX2
% ---
\usepackage{lipsum}				% para geração de dummy text
% ---
		
% ---
% Pacotes de citações
% ---
\usepackage[brazilian,hyperpageref]{backref}	 % Paginas com as citações na bibl
\usepackage[alf]{abntex2cite}	% Citações padrão ABNT
% ---

% ---
% Configurações do pacote backref
% Usado sem a opção hyperpageref de backref
\renewcommand{\backrefpagesname}{Citado na(s) página(s):~}
% Texto padrão antes do número das páginas
\renewcommand{\backref}{}
% Define os textos da citação
\renewcommand*{\backrefalt}[4]{
	\ifcase #1 %
		Nenhuma citação no texto.%
	\or
		Citado na página #2.%
	\else
		Citado #1 vezes nas páginas #2.%
	\fi}%
% ---

% --- Informações de dados para CAPA e FOLHA DE ROSTO ---
\title{Modelo Canônico de Artigo científico com \abnTeX}

\author{
Myke Albuquerque Pinto de Oliveira}

\local{Brasil}
\data{2018, v<VERSION>}
% ---

% ---
% Configurações de aparência do PDF final

% alterando o aspecto da cor azul
\definecolor{blue}{RGB}{41,5,195}

% informações do PDF
\makeatletter
\hypersetup{
     	%pagebackref=true,
		pdftitle={\@title}, 
		pdfauthor={\@author},
    	pdfsubject={Modelo de artigo científico com abnTeX2},
	    pdfcreator={LaTeX with abnTeX2},
		pdfkeywords={abnt}{latex}{abntex}{abntex2}{atigo científico}, 
		colorlinks=true,       		% false: boxed links; true: colored links
    	linkcolor=blue,          	% color of internal links
    	citecolor=blue,        		% color of links to bibliography
    	filecolor=magenta,      		% color of file links
		urlcolor=blue,
		bookmarksdepth=4
}
\makeatother
% --- 

% ---
% compila o indice
% ---
\makeindex
% ---

% ---
% Altera as margens padrões
% ---
\setlrmarginsandblock{3cm}{3cm}{*}
\setulmarginsandblock{3cm}{3cm}{*}
\checkandfixthelayout
% ---

% --- 
% Espaçamentos entre linhas e parágrafos 
% --- 

% O tamanho do parágrafo é dado por:
\setlength{\parindent}{1.3cm}

% Controle do espaçamento entre um parágrafo e outro:
\setlength{\parskip}{0.2cm}  % tente também \onelineskip

% Espaçamento simples
\SingleSpacing


% ----
% Início do documento
% ----
\begin{document}

% Seleciona o idioma do documento (conforme pacotes do babel)
%\selectlanguage{english}
\selectlanguage{brazil}

% Retira espaço extra obsoleto entre as frases.
\frenchspacing 

% ----------------------------------------------------------
% ELEMENTOS PRÉ-TEXTUAIS
% ----------------------------------------------------------

%---
%
% Se desejar escrever o artigo em duas colunas, descomente a linha abaixo
% e a linha com o texto ``FIM DE ARTIGO EM DUAS COLUNAS''.
% \twocolumn[    		% INICIO DE ARTIGO EM DUAS COLUNAS
%
%---

% página de titulo principal (obrigatório)
\maketitle


% titulo em outro idioma (opcional)



% resumo em português
\begin{resumoumacoluna}
 Conforme a ABNT NBR 6022:2018, o resumo no idioma do documento é elemento obrigatório. 
 Constituído de uma sequência de frases concisas e objetivas e não de uma 
 simples enumeração de tópicos, não ultrapassando 250 palavras, seguido, logo 
 abaixo, das palavras representativas do conteúdo do trabalho, isto é, 
 palavras-chave e/ou descritores, conforme a NBR 6028. (\ldots) As 
 palavras-chave devem figurar logo abaixo do resumo, antecedidas da expressão 
 Palavras-chave:, separadas entre si por ponto e finalizadas também por ponto.
 
 \vspace{\onelineskip}
 
 \noindent
 \textbf{Palavras-chave}: latex. abntex. editoração de texto.
\end{resumoumacoluna}


% resumo em inglês
\renewcommand{\resumoname}{Abstract}
\begin{resumoumacoluna}
 \begin{otherlanguage*}{english}
   According to ABNT NBR 6022:2018, an abstract in foreign language is optional.

   \vspace{\onelineskip}
 
   \noindent
   \textbf{Keywords}: latex. abntex.
 \end{otherlanguage*}  
\end{resumoumacoluna}

% ]  				% FIM DE ARTIGO EM DUAS COLUNAS
% ---

\begin{center}\smaller
\textbf{Data de submissão e aprovação}: elemento obrigatório. Indicar dia, mês e ano

\textbf{Identificação e disponibilidade}: elemento opcional. Pode ser indicado 
o endereço eletrônico, DOI, suportes e outras informações relativas ao acesso.
\end{center}

% ----------------------------------------------------------
% ELEMENTOS TEXTUAIS
% ----------------------------------------------------------
\textual

% ----------------------------------------------------------
% Introdução
% ----------------------------------------------------------
\section{Introdução}

Este documento e seu código-fonte são exemplos de referência de uso da classe
\textsf{abntex2} e do pacote \textsf{abntex2cite}. O documento exemplifica a
elaboração de publicação periódica científica impressa produzida conforme a ABNT
NBR 6022:2018 \emph{Informação e documentação - Artigo em publicação periódica
científica - Apresentação}.

A expressão ``Modelo canônico'' é utilizada para indicar que \abnTeX\ não é
modelo específico de nenhuma universidade ou instituição, mas que implementa tão
somente os requisitos das normas da ABNT. Uma lista completa das normas
observadas pelo \abnTeX\ é apresentada em \citeonline{abntex2classe}.

Sinta-se convidado a participar do projeto \abnTeX! Acesse o site do projeto em
\url{http://www.abntex.net.br/}. Também fique livre para conhecer,
estudar, alterar e redistribuir o trabalho do \abnTeX, desde que os arquivos
modificados tenham seus nomes alterados e que os créditos sejam dados aos
autores originais, nos termos da ``The \LaTeX\ Project Public
License''\footnote{\url{http://www.latex-project.org/lppl.txt}}.

Encorajamos que sejam realizadas customizações específicas deste documento.
Porém, recomendamos que ao invés de se alterar diretamente os arquivos do
\abnTeX, distribua-se arquivos com as respectivas customizações. Isso permite
que futuras versões do \abnTeX~não se tornem automaticamente incompatíveis com
as customizações promovidas. Consulte \citeonline{abntex2-wiki-como-customizar}
para mais informações.

Este exemplo deve ser utilizado como complemento do manual da classe
\textsf{abntex2} \cite{abntex2classe}, dos manuais do pacote
\textsf{abntex2cite} \cite{abntex2cite,abntex2cite-alf} e do manual da classe
\textsf{memoir} \cite{memoir}. Consulte o \citeonline{abntex2modelo} para obter
exemplos e informações adicionais de uso de \abnTeX\ e de \LaTeX.

% ----------------------------------------------------------
% Seção de explicações
% ----------------------------------------------------------
\section{Exemplos de comandos}

\subsection{Margens}

A norma ABNT NBR 6022:2018 não estabelece uma margem específica a ser utilizada
no artigo científico. Dessa maneira, caso deseje alterar as margens, utilize os
comandos abaixo:

\begin{verbatim}
   \setlrmarginsandblock{3cm}{3cm}{*}
   \setulmarginsandblock{3cm}{3cm}{*}
   \checkandfixthelayout
\end{verbatim}

\subsection{Duas colunas}

É comum que artigos científicos sejam escritos em duas colunas. Para isso,
adicione a opção \texttt{twocolumn} à classe do documento, como no exemplo:

\begin{verbatim}
   \documentclass[article,11pt,oneside,a4paper,twocolumn]{abntex2}
\end{verbatim}

É possível indicar pontos do texto que se deseja manter em apenas uma coluna,
geralmente o título e os resumos. Os resumos em única coluna em documentos com
a opção \texttt{twocolumn} devem ser escritos no ambiente
\texttt{resumoumacoluna}:

\begin{verbatim}
   \twocolumn[              % INICIO DE ARTIGO EM DUAS COLUNAS

     \maketitle             % pagina de titulo

     \renewcommand{\resumoname}{Nome do resumo}
     \begin{resumoumacoluna}
        Texto do resumo.
      
        \vspace{\onelineskip}
 
        \noindent
        \textbf{Palavras-chave}: latex. abntex. editoração de texto.
     \end{resumoumacoluna}
   
   ]                        % FIM DE ARTIGO EM DUAS COLUNAS
\end{verbatim}

\subsection{Recuo do ambiente \texttt{citacao}}

Na produção de artigos (opção \texttt{article}), pode ser útil alterar o recuo
do ambiente \texttt{citacao}. Nesse caso, utilize o comando:

\begin{verbatim}
   \setlength{\ABNTEXcitacaorecuo}{1.8cm}
\end{verbatim}

Quando um documento é produzido com a opção \texttt{twocolumn}, a classe
\textsf{abntex2} automaticamente altera o recuo padrão de 4 cm, definido pela
ABNT NBR 10520:2002 seção 5.3, para 1.8 cm.

\section{Cabeçalhos e rodapés customizados}

Diferentes estilos de cabeçalhos e rodapés podem ser criados usando os
recursos padrões do \textsf{memoir}.

Um estilo próprio de cabeçalhos e rodapés pode ser diferente para páginas pares
e ímpares. Observe que a diferenciação entre páginas pares e ímpares só é
utilizada se a opção \texttt{twoside} da classe \textsf{abntex2} for utilizado.
Caso contrário, apenas o cabeçalho padrão da página par (\emph{even}) é usado.

Veja o exemplo abaixo cria um estilo chamado \texttt{meuestilo}. O código deve
ser inserido no preâmbulo do documento.

\begin{verbatim}
%%criar um novo estilo de cabeçalhos e rodapés
\makepagestyle{meuestilo}
  %%cabeçalhos
  \makeevenhead{meuestilo} %%pagina par
     {topo par à esquerda}
     {centro \thepage}
     {direita}
  \makeoddhead{meuestilo} %%pagina ímpar ou com oneside
     {topo ímpar/oneside à esquerda}
     {centro\thepage}
     {direita}
  \makeheadrule{meuestilo}{\textwidth}{\normalrulethickness} %linha
  %% rodapé
  \makeevenfoot{meuestilo}
     {rodapé par à esquerda} %%pagina par
     {centro \thepage}
     {direita} 
  \makeoddfoot{meuestilo} %%pagina ímpar ou com oneside
     {rodapé ímpar/onside à esquerda}
     {centro \thepage}
     {direita}
\end{verbatim}

Para usar o estilo criado, use o comando abaixo imediatamente após um dos
comandos de divisão do documento. Por exemplo:

\begin{verbatim}
   \begin{document}
     %%usar o estilo criado na primeira página do artigo:
     \pretextual
     \pagestyle{meuestilo}
     
     \maketitle
     ...
     
     %%usar o estilo criado nas páginas textuais
     \textual
     \pagestyle{meuestilo}
     
     \chapter{Novo capítulo}
     ...
   \end{document}  
\end{verbatim}
   
Outras informações sobre cabeçalhos e rodapés estão disponíveis na seção 7.3 do
manual do \textsf{memoir} \cite{memoir}.

\section{Mais exemplos no Modelo Canônico de Trabalhos Acadêmicos}

Este modelo de artigo é limitado em número de exemplos de comandos, pois são
apresentados exclusivamente comandos diretamente relacionados com a produção de
artigos.

Para exemplos adicionais de \abnTeX\ e \LaTeX, como inclusão de figuras,
fórmulas matemáticas, citações, e outros, consulte o documento
\citeonline{abntex2modelo}.

\section{Consulte o manual da classe \textsf{abntex2}}

Consulte o manual da classe \textsf{abntex2} \cite{abntex2classe} para uma
referência completa das macros e ambientes disponíveis.

% ---
% Finaliza a parte no bookmark do PDF, para que se inicie o bookmark na raiz
% ---
\bookmarksetup{startatroot}% 
% ---

% ---
% Conclusão
% ---
\section{Considerações finais}

\lipsum[1]

\begin{citacao}
\lipsum[2]
\end{citacao}

\lipsum[3]

% ----------------------------------------------------------
% ELEMENTOS PÓS-TEXTUAIS
% ----------------------------------------------------------
\postextual

% ----------------------------------------------------------
% Referências bibliográficas
% ----------------------------------------------------------
\bibliography{abntex2-modelo-references}

% ----------------------------------------------------------
% Glossário
% ----------------------------------------------------------
%
% Há diversas soluções prontas para glossário em LaTeX. 
% Consulte o manual do abnTeX2 para obter sugestões.
%
%\glossary

% ----------------------------------------------------------
% Apêndices
% ----------------------------------------------------------

% ---
% Inicia os apêndices
% ---
\begin{apendicesenv}

% ----------------------------------------------------------
\chapter{Nullam elementum urna vel imperdiet sodales elit ipsum pharetra ligula
ac pretium ante justo a nulla curabitur tristique arcu eu metus}
% ----------------------------------------------------------
\lipsum[55-56]

\end{apendicesenv}
% ---

% ----------------------------------------------------------
% Anexos
% ----------------------------------------------------------
\cftinserthook{toc}{AAA}
% ---
% Inicia os anexos
% ---
%\anexos
\begin{anexosenv}

% ---
\chapter{Cras non urna sed feugiat cum sociis natoque penatibus et magnis dis
parturient montes nascetur ridiculus mus}
% ---

\lipsum[31]

\end{anexosenv}

% ----------------------------------------------------------
% Agradecimentos
% ----------------------------------------------------------

\section*{Agradecimentos}
Texto sucinto aprovado pelo periódico em que será publicado. Último 
elemento pós-textual.

\end{document}
